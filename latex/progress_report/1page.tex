\documentclass{article}

\usepackage{comment}
\usepackage[english]{isodate}
\usepackage{graphicx}
\usepackage[margin=1in]{geometry}
\usepackage{siunitx}
\usepackage{paracol}
\usepackage{enumitem}
\usepackage{bbm}
\usepackage[utf8]{inputenc}
\usepackage[T1]{fontenc}
\usepackage[bookmarks=true]{hyperref}
\usepackage{bookmark}
\usepackage{pdfpages} %\includepdf[pages={1}]{myfile.pdf}

\usepackage{amsmath}
\usepackage{ amssymb }
\usepackage{amsthm}
\usepackage{mdsymbol}%for perp with two bars

\usepackage{mathtools,xparse}
\newtheorem{theorem}{Theorem}

\DeclarePairedDelimiter{\abs}{\lvert}{\rvert}
\DeclarePairedDelimiter{\norm}{\lVert}{\rVert}

\newcommand{\E}{\mathbb{E}}
\newcommand{\Var}{\mathrm{Var}}
\newcommand{\Cov}{\mathrm{Cov}}
\newcommand\given[1][]{\:#1\vert\:}


\sisetup{output-decimal-marker = {,}}
\newcommand*{\ft}[1]{_\mathrm{#1}} 
\newcommand*{\dd}{\mathop{}\!\mathrm{d}}
\newcommand*{\tran}{^{\mkern-1.5mu\mathsf{T}}}%transpose of matrix
\newcommand{\trace}{\mathrm{trace}}

%%new
\newcommand{\tab}{\hspace{.2\textwidth}}
%\newcommand{\span}{\mathrm{Span}}
\renewcommand{\baselinestretch}{1.5}



%%%indenting
\newlength\tindent
\setlength{\tindent}{\parindent}
\setlength{\parindent}{0pt}
\renewcommand{\indent}{\hspace*{\tindent}}


\usepackage{titling}
\usepackage{fancyhdr}
\pagestyle{fancy}
\title{Synthetic Text Generation - Progress Report}

\author{Jimmy Leroux, Frédéric Boileau, Nicolas Laliberté }
\date{\today}

\begin{document}

\fancyhf{}
\fancyhead[L]{ IFT6269-A2018, Prof Simon Lacoste-Julien, group 19}
\maketitle
\thispagestyle{fancy}

\begin{abstract}
In our original project plan we had indicated that we wished to explore
different network topologies to build a generative model for time series data,
more specifically short text sequences. After parsing through the latest
literature we have opted for a Long-Short Term Memory (LSTM) network. They are
several reasons for this. The most simple option would have been to use a
Hidden Markov Model (HMM), however this topology suffers from several
drawbacks, first the Markovian independence  assumption is unreasonably strong
for most text processing and using an N'th order HMM is not an alternative as
time complexity grows exponentially with N. The Recurrent Neural Network (RNN)
type of neural network allows one to introduce memory into an Artifical Neural
Network (ANN) while keeping the model tractable. However RNN's suffer from
numerical issues, most notably the vanishing-exploding gradient problem which
is adressed by using using the LSTM configuration. \\
\end{abstract}
\clearpage

We have decided to build a quote generator with LSTM architecture,
built using the Pytorch library, the later being state of the art in many 
domains, intuitive, and providing easy access to GPU acceleration.  The dataset
we used for the training is from Kaggle and is composed of 36.2k quotes
\cite{quote}. First, we build our dictionary by extracting all the different
words from our dataset. We then create an embedding layer which is responsible
to learn a vector representation for each of our words. We have integrated the
possibility to initialize our embedding layer with a pretrained GloVe model if
we which to \cite{glove}. We are feeding sequences of 4096 encoded words to
4096 LSTM units with a hidden state of 100 dimensions. The output of each LSTM
unit is then fed in a hidden layer with RELU activation functions and finally 
passing through a softmax activation for the identification of the predictions
at the output. We are using the \textit{cross-entropy loss as objective to optimize},
thus maximizing the loglikelihood of the training data. For the optimization
technique, we \textit{used the Adam optimizer}, since it seems to be the state of the
art now \cite{adam}. To generate text, we initialize the network with a seed
word, in our case it was 'What', and then feed back the predictions in the
network in loops. To incorporate randomness in the generated text, we choose
the predicted word by defining a multinomial distribution on the output of the
softmax layer. Here's an exemple of generated text: "What is the best thing
that has arises from me is coming out of mental exaltation." \\

In the following weeks we plan to explore different ways to make the model
perform better. Now that we are confident in the general configuration of our
model (LSTM) we can experiment with the cost function, the activation functions
(which may differ per layer) the depth of the model, the number of hidden units
and the general optimization procedure. Even a choice of optimization procedure
leads to other questions for tuning our model as they are themselves governed
by hyperparameter. If time allows we will also compare the main options for the vector
encoding of words, the two contenders being Glove2Vec and word2vec.
   
Finally we will consider other tasks than the generation of synthetic text
and evaluate what changes would need to be performed to tune it for those
other NLP applications.

\clearpage
\bibliographystyle{unsrt}
\bibliography{../biblio} 
\end{document}
