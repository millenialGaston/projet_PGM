% For math formating
\usepackage{amsmath}




\begin{document}
\section{Implementation}
In our original project plan we had indicated that we wished to explore 
different network topologies to build a generative model for time series 
data, more specifically short text sequences. After parsing through the latest
literature we have opted for a RNN-LSTM network. They are several reasons for
this. The most simple option would have been to use a Hidden Markov Model
(HMM), however this topology suffers from several drawbacks, first the 
Markov assumption is unreasonably strong for most text processing and 
using an N'th order HMM is not an alternative as time complexity grows 
exponentially with N. The Recurrent Neural Network (RNN) type of neural
network allows one to introduce memory into an Artifical Neural Network (ANN)
while keeping the model tractable. However RNN's suffer from numerical issues,
most notably the vanishing-exploding gradient problem which is readily solved
using a Long-Short Term Model (LSTM). \\

For the implementation part of our project, we have decided to make a quote
generator using RNNs. For this implementation, we used the Pytorch library
because we found it more intuitive and easier to debug then other frameworks.
The dataset we used for the training is from Kaggle and is composed of 36.2k
quotes \cite{quote}. First, we build our dictionary by extracting all the
different words from our dataset. We then create an embedding layer which is
responsible to learn a vector representation for each of our words. We have
integrated the possibility to initialize our embedding layer with a pretrained
GloVe model if we which to \cite{glove}. We are feeding sequences of 4096
encoded words to 4096 LSTM units with a hidden state of 100 dimensions. The
output of each LSTM unit is then fed in a linear layer before passing through a
softmax activation for the identification of the predictions. We are using the
cross-entropy loss as objective to optimize, thus maximizing the loglikelihood
of the training data. For the optimization technique, we used the Adam
optimizer, since it seems to be the state of the art now \cite{adam}. To
generate text, we initialize the network with a seed word, in our case it was
'What', and then feed back the predictions in the network in loops. To
incorporate randomness in the generated text, we choose the predicted word by
defining a multinomial distribution on the output of the softmax layer. Here's
an exemple of generated text: "What is the best thing that has arises from me
is coming out of mental exaltation."    

\clearpage
\bibliographystyle{unsrt}
\bibliography{../biblio} 
\end{document}
