\documentclass{article}

\usepackage{comment}
\usepackage[english]{isodate}
\usepackage{graphicx}
\usepackage[margin=1in]{geometry}
\usepackage{siunitx}
\usepackage{paracol}
\usepackage{enumitem}
\usepackage{bbm}
\usepackage[utf8]{inputenc}
\usepackage[T1]{fontenc}
\usepackage[bookmarks=true]{hyperref}
\usepackage{bookmark}
\usepackage{pdfpages} %\includepdf[pages={1}]{myfile.pdf}

\usepackage{amsmath}
\usepackage{ amssymb }
\usepackage{amsthm}
\usepackage{mdsymbol}%for perp with two bars

\usepackage{mathtools,xparse}
\newtheorem{theorem}{Theorem}

\DeclarePairedDelimiter{\abs}{\lvert}{\rvert}
\DeclarePairedDelimiter{\norm}{\lVert}{\rVert}

\newcommand{\E}{\mathbb{E}}
\newcommand{\Var}{\mathrm{Var}}
\newcommand{\Cov}{\mathrm{Cov}}
\newcommand\given[1][]{\:#1\vert\:}


\sisetup{output-decimal-marker = {,}}
\newcommand*{\ft}[1]{_\mathrm{#1}} 
\newcommand*{\dd}{\mathop{}\!\mathrm{d}}
\newcommand*{\tran}{^{\mkern-1.5mu\mathsf{T}}}%transpose of matrix
\newcommand{\trace}{\mathrm{trace}}

%%new
\newcommand{\tab}{\hspace{.2\textwidth}}
%\newcommand{\span}{\mathrm{Span}}
\renewcommand{\baselinestretch}{1.5}



%%%indenting
\newlength\tindent
\setlength{\tindent}{\parindent}
\setlength{\parindent}{0pt}
\renewcommand{\indent}{\hspace*{\tindent}}


\begin{document}

In our original project plan we had indicated that we wished to explore 
different network topologies to build a generative model for time series 
data, more specifically short text sequences. After parsing through the latest
literature we have opted for a RNN-LSTM network. They are several reasons for
this. The most simple option would have been to use a Hidden Markov Model
(HMM), however this topology suffers from several drawbacks, first the 
Markov assumption is unreasonably strong for most text processing and 
using an N'th order HMM is not an alternative as time complexity grows 
exponentially with N. The Recurrent Neural Network (RNN) type of neural
network allows one to introduce memory into an Artifical Neural Network (ANN)
while keeping the model tractable. However RNN's suffer from numerical issues,
most notably the vanishing-exploding gradient problem which is readily solved
using a Long-Short Term Model (LSTM). \\

More concretely we have decided to build a quote generator with LSTM architecture,
built using the Pytorch library, the later being state of the art in many 
domains, intuitive, and providing easy access to GPU acceleration.  The dataset
we used for the training is from Kaggle and is composed of 36.2k quotes
\cite{quote}. First, we build our dictionary by extracting all the different
words from our dataset. We then create an embedding layer which is responsible
to learn a vector representation for each of our words. We have integrated the
possibility to initialize our embedding layer with a pretrained GloVe model if
we which to \cite{glove}. We are feeding sequences of 4096 encoded words to
4096 LSTM units with a hidden state of 100 dimensions. The output of each LSTM
unit is then fed in a linear layer before passing through a softmax activation
for the identification of the predictions. We are using the cross-entropy loss
as objective to optimize, thus maximizing the loglikelihood of the training
data. For the optimization technique, we used the Adam optimizer, since it
seems to be the state of the art now \cite{adam}. To generate text, we
initialize the network with a seed word, in our case it was 'What', and then
feed back the predictions in the
network in loops. To incorporate randomness in the generated text, we choose
the predicted word by defining a multinomial distribution on the output of the
softmax layer. Here's an exemple of generated text: "What is the best thing
that has arises from me is coming out of mental exaltation."    

\clearpage
\bibliographystyle{unsrt}
\bibliography{../biblio} 
\end{document}
